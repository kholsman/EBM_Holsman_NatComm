\documentclass[]{article}
\usepackage{lmodern}
\usepackage{amssymb,amsmath}
\usepackage{ifxetex,ifluatex}
\usepackage{fixltx2e} % provides \textsubscript
\ifnum 0\ifxetex 1\fi\ifluatex 1\fi=0 % if pdftex
  \usepackage[T1]{fontenc}
  \usepackage[utf8]{inputenc}
\else % if luatex or xelatex
  \ifxetex
    \usepackage{mathspec}
  \else
    \usepackage{fontspec}
  \fi
  \defaultfontfeatures{Ligatures=TeX,Scale=MatchLowercase}
\fi
% use upquote if available, for straight quotes in verbatim environments
\IfFileExists{upquote.sty}{\usepackage{upquote}}{}
% use microtype if available
\IfFileExists{microtype.sty}{%
\usepackage{microtype}
\UseMicrotypeSet[protrusion]{basicmath} % disable protrusion for tt fonts
}{}
\usepackage[margin=1in]{geometry}
\usepackage[unicode=true]{hyperref}
\hypersetup{
            pdftitle={EBM Nature Communications main script},
            pdfauthor={Kirstin Holsman},
            pdfborder={0 0 0},
            breaklinks=true}
\urlstyle{same}  % don't use monospace font for urls
\usepackage{color}
\usepackage{fancyvrb}
\newcommand{\VerbBar}{|}
\newcommand{\VERB}{\Verb[commandchars=\\\{\}]}
\DefineVerbatimEnvironment{Highlighting}{Verbatim}{commandchars=\\\{\}}
% Add ',fontsize=\small' for more characters per line
\usepackage{framed}
\definecolor{shadecolor}{RGB}{248,248,248}
\newenvironment{Shaded}{\begin{snugshade}}{\end{snugshade}}
\newcommand{\KeywordTok}[1]{\textcolor[rgb]{0.13,0.29,0.53}{\textbf{{#1}}}}
\newcommand{\DataTypeTok}[1]{\textcolor[rgb]{0.13,0.29,0.53}{{#1}}}
\newcommand{\DecValTok}[1]{\textcolor[rgb]{0.00,0.00,0.81}{{#1}}}
\newcommand{\BaseNTok}[1]{\textcolor[rgb]{0.00,0.00,0.81}{{#1}}}
\newcommand{\FloatTok}[1]{\textcolor[rgb]{0.00,0.00,0.81}{{#1}}}
\newcommand{\ConstantTok}[1]{\textcolor[rgb]{0.00,0.00,0.00}{{#1}}}
\newcommand{\CharTok}[1]{\textcolor[rgb]{0.31,0.60,0.02}{{#1}}}
\newcommand{\SpecialCharTok}[1]{\textcolor[rgb]{0.00,0.00,0.00}{{#1}}}
\newcommand{\StringTok}[1]{\textcolor[rgb]{0.31,0.60,0.02}{{#1}}}
\newcommand{\VerbatimStringTok}[1]{\textcolor[rgb]{0.31,0.60,0.02}{{#1}}}
\newcommand{\SpecialStringTok}[1]{\textcolor[rgb]{0.31,0.60,0.02}{{#1}}}
\newcommand{\ImportTok}[1]{{#1}}
\newcommand{\CommentTok}[1]{\textcolor[rgb]{0.56,0.35,0.01}{\textit{{#1}}}}
\newcommand{\DocumentationTok}[1]{\textcolor[rgb]{0.56,0.35,0.01}{\textbf{\textit{{#1}}}}}
\newcommand{\AnnotationTok}[1]{\textcolor[rgb]{0.56,0.35,0.01}{\textbf{\textit{{#1}}}}}
\newcommand{\CommentVarTok}[1]{\textcolor[rgb]{0.56,0.35,0.01}{\textbf{\textit{{#1}}}}}
\newcommand{\OtherTok}[1]{\textcolor[rgb]{0.56,0.35,0.01}{{#1}}}
\newcommand{\FunctionTok}[1]{\textcolor[rgb]{0.00,0.00,0.00}{{#1}}}
\newcommand{\VariableTok}[1]{\textcolor[rgb]{0.00,0.00,0.00}{{#1}}}
\newcommand{\ControlFlowTok}[1]{\textcolor[rgb]{0.13,0.29,0.53}{\textbf{{#1}}}}
\newcommand{\OperatorTok}[1]{\textcolor[rgb]{0.81,0.36,0.00}{\textbf{{#1}}}}
\newcommand{\BuiltInTok}[1]{{#1}}
\newcommand{\ExtensionTok}[1]{{#1}}
\newcommand{\PreprocessorTok}[1]{\textcolor[rgb]{0.56,0.35,0.01}{\textit{{#1}}}}
\newcommand{\AttributeTok}[1]{\textcolor[rgb]{0.77,0.63,0.00}{{#1}}}
\newcommand{\RegionMarkerTok}[1]{{#1}}
\newcommand{\InformationTok}[1]{\textcolor[rgb]{0.56,0.35,0.01}{\textbf{\textit{{#1}}}}}
\newcommand{\WarningTok}[1]{\textcolor[rgb]{0.56,0.35,0.01}{\textbf{\textit{{#1}}}}}
\newcommand{\AlertTok}[1]{\textcolor[rgb]{0.94,0.16,0.16}{{#1}}}
\newcommand{\ErrorTok}[1]{\textcolor[rgb]{0.64,0.00,0.00}{\textbf{{#1}}}}
\newcommand{\NormalTok}[1]{{#1}}
\usepackage{graphicx,grffile}
\makeatletter
\def\maxwidth{\ifdim\Gin@nat@width>\linewidth\linewidth\else\Gin@nat@width\fi}
\def\maxheight{\ifdim\Gin@nat@height>\textheight\textheight\else\Gin@nat@height\fi}
\makeatother
% Scale images if necessary, so that they will not overflow the page
% margins by default, and it is still possible to overwrite the defaults
% using explicit options in \includegraphics[width, height, ...]{}
\setkeys{Gin}{width=\maxwidth,height=\maxheight,keepaspectratio}
\IfFileExists{parskip.sty}{%
\usepackage{parskip}
}{% else
\setlength{\parindent}{0pt}
\setlength{\parskip}{6pt plus 2pt minus 1pt}
}
\setlength{\emergencystretch}{3em}  % prevent overfull lines
\providecommand{\tightlist}{%
  \setlength{\itemsep}{0pt}\setlength{\parskip}{0pt}}
\setcounter{secnumdepth}{0}
% Redefines (sub)paragraphs to behave more like sections
\ifx\paragraph\undefined\else
\let\oldparagraph\paragraph
\renewcommand{\paragraph}[1]{\oldparagraph{#1}\mbox{}}
\fi
\ifx\subparagraph\undefined\else
\let\oldsubparagraph\subparagraph
\renewcommand{\subparagraph}[1]{\oldsubparagraph{#1}\mbox{}}
\fi

\title{EBM Nature Communications main script}
\author{Kirstin Holsman}
\date{2/17/2020}

\begin{document}
\maketitle

\emph{Data and code are under review and subject to change. Do not use
without permission from lead author:
\href{mailto:kirstin.holsman@noaa.gov}{\nolinkurl{kirstin.holsman@noaa.gov}}}

\begin{center}\rule{0.5\linewidth}{0.5pt}\end{center}

\begin{itemize}
\tightlist
\item
  Kirstin Holsman\\
  Alaska Fisheries Science Center\\
  NOAA Fisheries, Seattle WA\\
  \textbf{\href{mailto:kirstin.holsman@noaa.gov}{\nolinkurl{kirstin.holsman@noaa.gov}}}\\
  \emph{Last updated: 2020} ---
\end{itemize}

\section{Overview}\label{overview}

This is an overview of the data, code, and workflow used to generate
intermediate and final data for the Holsman et al. in review Nature
Communications paper.

\subsection{Primary Data sources and
access:}\label{primary-data-sources-and-access}

Various simulation outputs were made available for use in this analysis
through the interdisciplinary
\href{\%22https://www.fisheries.noaa.gov/alaska/ecosystems/alaska-climate-integrated-modeling-project\%22}{Alaska
Climate Integrated Modeling (ACLIM) project}. An overview of the project
and simulation experiments can be found in Hollowed et al. 2020.

\subsubsection{ROMSNPZ}\label{romsnpz}

Downscaled hindcasts and CMIP5 projections of oceanogrpahic and
lowertrophic conditions were developed as part of the ACLIM project. An
overview of these projections and the Bering10K ROMSNPZ project can be
found in Hermann et al. 2019, Kearney et al. 2020, and Hollowed et al.
2020.

\subsubsection{CEATTLE Model}\label{ceattle-model}

CEATTLE is a multispecies stock assessment model that has been updated
annually and included as an appendix to the walleye pollock stock
assessment since 2016 as part of the Bering Sea fishery stock assessment
process. As part of ACLIM CEATTLE was coupled to the ROMSNPZ model and
the ATTACH model (below) to generate projections of species biomass and
catch under future climate conditions in the Bering Sea. Methods for
this coupling and projection simulation can be found in Holsman et al.
submitted and Hollowed et al. 2020.

\subsubsection{ATTACH Model}\label{attach-model}

The ATTACH model

\subsection{Intermediate data:}\label{intermediate-data}

Intermediate data can be found in the main EBM\_Holsman\_NatComm in the
form of .Rdata files but can be recreated (although this is not
recommended; see below) from the ADMB model using the
EBM\_Holsman\_NatComm/assessment\_scripts/README\_EBM\_Holsman\_Analysis.pdf.

\subsection{Figures and tables:}\label{figures-and-tables}

Final figures and tables (including illustrator files that were used to
add fish icons) can be found in the \textbf{Figures} folder.

\subsection{Running the analyses}\label{running-the-analyses}

Below are instructions for downloading input data, code and scripts to
recreate the figures, tables, results, and run the risk and threshold
analyses for the paper.

\subsubsection{Download input data from
figshare:}\label{download-input-data-from-figshare}

To run the analyses or create the paper figures you will need to first
download the large zipped data folder here:
\url{https://figshare.com/s/6dea7722df39e07d79f0} (Data
\url{DOI:10.6084/m9.figshare.11864505}) and copy - paste the contents
(folders ``in'' and ``out''``) it in the directory: {[}your local
directory path{]}/EBM\_Holsman\_NatComm/data

If you plan to use the data within the folder for purposes beyond
rerunning the paper analyses and figures please contact
\href{mailto:kirstin.holsman@noaa.gov}{\nolinkurl{kirstin.holsman@noaa.gov}}
and provide the cite the following Data
\url{DOI:10.6084/m9.figshare.11864505} along with Holsman et al. 2020.

\begin{Shaded}
\begin{Highlighting}[]
    \NormalTok{url        <-}\StringTok{ "https://figshare.com/s/6dea7722df39e07d79f0"}
    \NormalTok{dest_path  <-}\StringTok{ "/Users/kholsman/GitHub_new/EBM_Holsman_NatComm/EBM_ceattlenew.Rdata"}
    \CommentTok{# Apply download.file function in R}
    \KeywordTok{download.file}\NormalTok{(}\DataTypeTok{url=}\NormalTok{url, }\DataTypeTok{destfile=}\NormalTok{dest_path,}\DataTypeTok{method=}\StringTok{"libcurl"}\NormalTok{)}
\end{Highlighting}
\end{Shaded}

\subsubsection{Re-generate plots}\label{re-generate-plots}

If running plotting code below (recommended) you will need to download
the final data ``EBM\_ceattlenew.Rdata'' from figshare and place it in
the main directory: ``EBM\_Holsman\_NatComm/EBM\_ceattlenew.Rdata''.

\begin{itemize}
\tightlist
\item
  access EBM\_ceattlenew.Rdata here:
  \url{https://figshare.com/s/6dea7722df39e07d79f0} and place it in the
  directory: EBM\_Holsman\_NatComm/EBM\_ceattlenew.Rdata. (Data
  \url{DOI:10.6084/m9.figshare.11864505})
\end{itemize}

\subsubsection{Re-running the intermediate
data}\label{re-running-the-intermediate-data}

If re-running the intermediate data analysis (not recommended) the
following files will need to be downloaded unzipped and placed in the
asssesment\_files folder:

\begin{itemize}
\item
  access aclim\_00\_JunV2\_2019\_2.zip here:
  \url{https://figshare.com/s/3a1aaa86837b79d6aa07} and place it in the
  EBM\_Holsman\_NatComm/data/runs/aclim\_00\_JunV2\_2019\_2.zip and
  unzip. (Data \url{DOI:10.6084/m9.figshare.11864586})
\item
  access aclim\_00\_JunV2\_2019\_0.zip here:
  \url{https://figshare.com/s/d9c35dbe0880f4169041} and place it in the
  EBM\_Holsman\_NatComm/data/runs/aclim\_00\_JunV2\_2019\_0.zip and
  unzip (Data \url{DOI:10.6084/m9.figshare.11864577})
\end{itemize}

This is the main script for running analysis and plotting results and
requires R version 3.5.3 (available at
\url{https://cran.r-project.org/bin/macosx/el-capitan/base/}). To update
the analysis using .Rdata outputs run the R() code below as is currently
configured. If you want to update the intermediate data, set ``readdat''
to TRUE in line 80 below. The CEATTLE stock assessment is also included
but requires AD Model builder (\url{http://www.admb-project.org}). To
run the assessment scripts (not recommended or tested outside of macOSX)
see ``README\_EBM\_Holsman\_Analysis.pdf''.

\subsubsection{EMB\_paper.R script:}\label{emb_paper.r-script}

\begin{Shaded}
\begin{Highlighting}[]
\NormalTok{## ------------------------------------------------}
\NormalTok{## plotting code for EBM paper }
\NormalTok{## Kirstin Holsman }
\NormalTok{## Feb 2020}
\NormalTok{## Kirstin.holsman@noaa.gov}
\NormalTok{## ------------------------------------------------}


\CommentTok{# 1. Set up}
\CommentTok{# 2. load data}
\CommentTok{# 3. make figures}

    \KeywordTok{rm}\NormalTok{(}\DataTypeTok{list=}\KeywordTok{ls}\NormalTok{())}
    \KeywordTok{graphics.off}\NormalTok{()}
    \KeywordTok{interactive}\NormalTok{()}
      
    \CommentTok{#-------------------------------------}
    \CommentTok{#   1. SET THINGS UP}
    \CommentTok{#------------------------------------- }
    \CommentTok{# set your local path:}
    \NormalTok{main  <-}\StringTok{  }\KeywordTok{path.expand}\NormalTok{(}\StringTok{"~/GitHub_new/EBM_Holsman_NatComm/"}\NormalTok{)}
    \KeywordTok{setwd}\NormalTok{(main)}
    
    \CommentTok{# load data, packages, setup, etc.}
    \KeywordTok{source}\NormalTok{(}\StringTok{"R/make.R"}\NormalTok{)}
    
    \CommentTok{# --------------------------------------------------------------}
    \CommentTok{# optional: Make the paper figures:}
    \CommentTok{# --------------------------------------------------------------}
    \KeywordTok{source}\NormalTok{(}\StringTok{"R/sub_scripts/make_plotsV2.R"}\NormalTok{)  }\CommentTok{# this will generate the paper figures without overwriting them}
    \CommentTok{# You can also call individual plots like this:}
    
    \KeywordTok{fig2}\NormalTok{()}
    \KeywordTok{fig3}\NormalTok{()}
    \KeywordTok{fig4}\NormalTok{()}
    \KeywordTok{fig5}\NormalTok{()}
    \KeywordTok{fig6}\NormalTok{()}
    \KeywordTok{figS1}\NormalTok{()}
    \KeywordTok{figS2}\NormalTok{()}
    \KeywordTok{figS3}\NormalTok{()}
    \KeywordTok{figS4}\NormalTok{()}
    \KeywordTok{figS5}\NormalTok{()}
    \KeywordTok{figS6}\NormalTok{()}
    
    \NormalTok{update.figs  <-}\StringTok{ }\OtherTok{TRUE}  
    \KeywordTok{stop}\NormalTok{(}\StringTok{"warning! this will overwrite existing figures in the Figures folder"}\NormalTok{)}
    \KeywordTok{source}\NormalTok{(}\StringTok{"R/sub_scripts/make_plotsV2.R"}\NormalTok{)}
    
    \NormalTok{update.figs  <-}\StringTok{ }\OtherTok{FALSE}   \CommentTok{# return this to it's default setting}
    
    
    \CommentTok{# --------------------------------------------------------------}
    \CommentTok{# optional: Rerun paper analyses including risk and threshold/tipping points}
    \CommentTok{# --------------------------------------------------------------}
    \CommentTok{# to re-run the paper anlyses set update.outputs = TRUE }
    \CommentTok{# (this is set to FALSE by defualt) in the R/setup.R script}
    
    \NormalTok{update.outputs  <-}\StringTok{ }\OtherTok{TRUE}  
    \KeywordTok{stop}\NormalTok{(}\StringTok{"warning! this will overwrite Rdata files in the data folder"}\NormalTok{)}
    \KeywordTok{source}\NormalTok{(}\StringTok{"R/sub_scripts/SUB_EBM_paper.R"}\NormalTok{)}
    
    \NormalTok{update.outputs  <-}\StringTok{ }\OtherTok{FALSE}  \CommentTok{# once complete set to FALSE and reload new datafiles:}
    \KeywordTok{source}\NormalTok{(}\StringTok{"R/make.R"}\NormalTok{)}
    
   
    \CommentTok{# see the workflow}
    \CommentTok{#vis_drake_graph(plan)}
\end{Highlighting}
\end{Shaded}

\end{document}
